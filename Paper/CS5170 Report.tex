%%%%%%%%%%%%%%%%%%%%%%%%%%%%%%%%%%%%%%%%%%%%%%%%%%%%%%%%%%%%%%%%%%%%%%%%

%%% LaTeX Template for AAMAS-2025 (based on sample-sigconf.tex)
%%% Prepared by the AAMAS-2025 Program Chairs based on the version from AAMAS-2025. 

%%%%%%%%%%%%%%%%%%%%%%%%%%%%%%%%%%%%%%%%%%%%%%%%%%%%%%%%%%%%%%%%%%%%%%%%

%%% Start your document with the \documentclass command.


%%% == IMPORTANT ==
%%% Use the first variant below for the final paper (including auithor information).
%%% Use the second variant below to anonymize your submission (no authoir information shown).
%%% For further information on anonymity and double-blind reviewing, 
%%% please consult the call for paper information
%%% https://aamas2025.org/index.php/conference/calls/submission-instructions-main-technical-track/

%%%% For anonymized submission, use this
%\documentclass[sigconf,anonymous]{aamas} 

%%%% For camera-ready, use this
\documentclass[sigconf]{aamas} 


%%% Load required packages here (note that many are included already).

\usepackage{balance} % for balancing columns on the final page

%%%%%%%%%%%%%%%%%%%%%%%%%%%%%%%%%%%%%%%%%%%%%%%%%%%%%%%%%%%%%%%%%%%%%%%%

%%% AAMAS-2025 copyright block (do not change!)

\makeatletter
\gdef\@copyrightpermission{
  \begin{minipage}{0.2\columnwidth}
   \href{https://creativecommons.org/licenses/by/4.0/}{\includegraphics[width=0.90\textwidth]{by}}
  \end{minipage}\hfill
  \begin{minipage}{0.8\columnwidth}
   \href{https://creativecommons.org/licenses/by/4.0/}{This work is licensed under a Creative Commons Attribution International 4.0 License.}
  \end{minipage}
  \vspace{5pt}
}
\makeatother

\setcopyright{ifaamas}
\acmConference[AAMAS '25]{Proc.\@ of the 24th International Conference
on Autonomous Agents and Multiagent Systems (AAMAS 2025)}{May 19 -- 23, 2025}
{Detroit, Michigan, USA}{Y.~Vorobeychik, S.~Das, A.~Nowé  (eds.)}
\copyrightyear{2025}
\acmYear{2025}
\acmDOI{}
\acmPrice{}
\acmISBN{}

%%%%%%%%%%%%%%%%%%%%%%%%%%%%%%%%%%%%%%%%%%%%%%%%%%%%%%%%%%%%%%%%%%%%%%%%

%%% == IMPORTANT ==
%%% Use this command to specify your OpenReview submission number.
%%% In anonymous mode, it will be printed on the first page.

\acmSubmissionID{<<OpenReview submission id>>}

%%% Use this command to specify the title of your paper.

\title[AAMAS-2025 Formatting Instructions]{Formatting Instructions for the 24th International Conference on Autonomous Agents and Multiagent Systems}

% Add the subtitle below for an extended abstract
%\subtitle{Extended Abstract}

%%% Provide names, affiliations, and email addresses for all authors.

\author{Kevin Tang}
\affiliation{
  \institution{Northeastern University}
  \city{Boston}
  \country{United States}}
\email{tang.kevi@northeastern.edu}

\author{Tanay}
\affiliation{
  \institution{Northeastern University}
  \city{Boston}
  \country{United States}}
\email{pandya.t@northeastern.edu}

\author{Name}
\affiliation{
  \institution{The Lady's Lake}
  \city{Avalon}
  \country{United Kingdom}}
\email{lady.of.the.lake@avalon.uk}

%%% Use this environment to specify a short abstract for your paper.

\begin{abstract}
Traditional learning methods often fail to capture the attention of the learner since there is a discrepancy between the ideas being taught and what the learner knows. This results in reduced engagement and a longer comprehension process for the learning materials. Our project aims to alleviate this issue through interactions with simulated historical figures. These figures were created with an emphasis on historical accuracy, distinctive speech patterns, and adaptive conversational behavior, allowing for a more natural learning experience. Our approach is unique from other work since we want to capture the historical figure as a whole with only access to what we know about them in history. Additionally, there is a focus on having a conversation verbally instead of solely being a chatbot. Preliminary results indicate that this method is not only more engaging but also helps users recall facts and historical context faster and more accurately. These early results are encouraging and support the potential of character-driven educational tools to transform how we learn.
\end{abstract}

%%% The code below was generated by the tool at http://dl.acm.org/ccs.cfm.
%%% Please replace this example with code appropriate for your own paper.


%%% Use this command to specify a few keywords describing your work.
%%% Keywords should be separated by commas.

\keywords{Artificial Intelligence, Human AI Interaction, RAG, LLM}

%%%%%%%%%%%%%%%%%%%%%%%%%%%%%%%%%%%%%%%%%%%%%%%%%%%%%%%%%%%%%%%%%%%%%%%%

%%% Include any author-defined commands here.
         
\newcommand{\BibTeX}{\rm B\kern-.05em{\sc i\kern-.025em b}\kern-.08em\TeX}

%%%%%%%%%%%%%%%%%%%%%%%%%%%%%%%%%%%%%%%%%%%%%%%%%%%%%%%%%%%%%%%%%%%%%%%%

\begin{document}

%%% The following commands remove the headers in your paper. For final 
%%% papers, these will be inserted during the pagination process.

\pagestyle{fancy}
\fancyhead{}

%%% The next command prints the information defined in the preamble.

\maketitle 

%%%%%%%%%%%%%%%%%%%%%%%%%%%%%%%%%%%%%%%%%%%%%%%%%%%%%%%%%%%%%%%%%%%%%%%%

\section{Introduction}

Current educational methods fail to engage users and provide them with the necessary context to fully grasp the information. As seen when studying history, textbooks and lectures fail to have a meaningful lasting impact. Additionally, often the actions of the past can be seen as contradictory or even controversial to things that we know now in modern day. Our project attempts to engage the user more thoughtfully by establishing a connection between the ancient and modern day ideas. We believe that by creating a virtual clone of a historical figure, it will boost engagement and understanding of history.

A more personalized learning experience can be cultivated with a conversation with a historical figure, however there were some challenges at hand. The first was to ensure that our historical figure was as accurate as possible to the original, which in our case was Pliny the Elder. This meant providing the clone with accurate information and writings from Pliny the Elder. Furthermore, the manner of speech needed to match the specific time period as well. To fix this, we had the LLM model mimic the writing of people within that time frame. Finally, to ensure that the model would accurately portray Pliny the Elder, prompt engineering was applied.

From here, we had three hypotheses we wanted to test. First, our system will be able to answer questions accurately with respect to the knowledge known by the historical figure. Second, our system will produce statements that are indistinguishable in tone from the source figure’s tone. Finally, our system will increase engagement and speed up learning among users. Initial surveys indicate increased engagement and comprehension, along with a faster learning process. Additionally, initial tests indicate our system demonstrates a high degree of both tonal authenticity and historical accuracy.


%%%%%%%%%%%%%%%%%%%%%%%%%%%%%%%%%%%%%%%%%%%%%%%%%%%%%%%%%%%%%%%%%%%%%%%%

\section{Background}


%%%%%%%%%%%%%%%%%%%%%%%%%%%%%%%%%%%%%%%%%%%%%%%%%%%%%%%%%%%%%%%%%%%%%%%%

\section{The Body of the Paper}

For help with typesetting the body of your paper in \LaTeX\@, please 
make use of the familiar resources~\cite{Lam94}. In this section we 
merely highlight a few specific features. 

\subsection{Mathematical Expressions}

You can typeset all sorts of in-line mathematical expressions 
with the usual \verb|$...$| construct, as in 
$\Diamond\Diamond\varphi \rightarrow \Diamond\varphi$ or 
$\boldsymbol{R} = (R_1,\ldots,R_n)$.
For more complex expressions, it may often be preferable to use one of
the various equation-type environments available in \LaTeX\@, as shown 
in the following example:
%
\begin{eqnarray}
Z_i & = & \frac{u_i(x_i) - u_i(x_{-i})}{u_i(x_i)}
\end{eqnarray}
%
Here is a second example for an equation: 
%
\begin{eqnarray}\label{eq:vcg}
p_i(\boldsymbol{\hat{v}}) & = &
\sum_{j \neq i} \hat{v}_j(f(\boldsymbol{\hat{v}}_{-i})) - 
\sum_{j \neq i} \hat{v}_j(f(\boldsymbol{\hat{v}})) 
\end{eqnarray}
%
Use the usual combination of `\verb|\label|' and `\verb|\ref|' to refer
to numbered equations, such as Equation~(\ref{eq:vcg}) above. Of course,
introducing numbers in the first place is only helpful if you in fact 
need to refer back to the equation in question elsewhere in the paper.


\subsection{Analysis}

Use the `\texttt{table}' environment (or its variant `\texttt{table*}')
in combination with the `\texttt{tabular}' environment to typeset tables
as floating objects. The `\texttt{aamas}' document class includes the 
`\texttt{booktabs}' package for preparing high-quality tables. Tables 
are often placed at the top of a page near their initial cite, as done 
here for Table~\ref{tab:locations}.

\begin{table}[t]
	\caption{Locations of the first five editions of AAMAS}
	\label{tab:locations}
	\begin{tabular}{rll}\toprule
		\textit{Year} & \textit{City} & \textit{Country} \\ \midrule
		2002 & Bologna & Italy \\
		2003 & Melbourne & Australia \\
		2004 & New York City & USA \\
		2005 & Utrecht & The Netherlands \\
		2006 & Hakodate & Japan \\ \bottomrule
	\end{tabular}
\end{table}

The caption of a table should be placed \emph{above} the table. 
Always use the `\verb|\midrule|' command to separate header rows from 
data rows, and use it only for this purpose. This enables assistive 
technologies to recognise table headers and support their users in 
navigating tables more easily.

%%% The following command should be issued somewhere in the first column 
%%% of the final page of your paper.
\balance

Use the `\texttt{figure}' environment for figures. If your figure 
contains third-party material, make sure to clearly identify it as such.
Every figure should include a caption, and this caption should be placed 
\emph{below} the figure itself, as shown here for Figure~\ref{fig:logo}.

\begin{figure}[h]
  \centering
  \includegraphics[width=0.75\linewidth]{aamas2025logo}
  \caption{The logo of AAMAS 2025}
  \label{fig:logo}
  \Description{Logo of AAMAS 2025 -- The 24th International Conference on Autonomous Agents and Multiagent Systems.}
\end{figure}

In addition, every figure should also have a figure description, unless
it is purely decorative. Use the `\verb|\Description|' command for this 
purpose. These descriptions will not be printed but can be used to 
convey what's in an image to someone who cannot see it. They are also 
used by search engine crawlers for indexing images, and when images 
cannot be loaded. A figure description must consist of unformatted plain 
text of up to 2000~characters. For example, the definition of 
Figure~\ref{fig:logo} in the source file of this document includes the 
following description: ``Logo of AAMAS 2025 -- The 24th International Conference on Autonomous Agents and Multiagent Systems.'' For more information on how best to write figure descriptions 
and why doing so is important, consult the information available here: 
%
\begin{center}
\url{https://www.acm.org/publications/taps/describing-figures/}
\end{center}
%
The use of colour in figures and graphs is permitted, provided they 
remain readable when printed in greyscale and provided they are 
intelligible also for people with a colour vision deficiency.

%%%%%%%%%%%%%%%%%%%%%%%%%%%%%%%%%%%%%%%%%%%%%%%%%%%%%%%%%%%%%%%%%%%%%%%%

\section{Citations and References}
  
The use of the \BibTeX\ to prepare your list of references is highly 
recommended. To include the references at the end of your document, put 
the following two commands just before the `\verb|\end{document}|' 
command in your source file:
%
\begin{verbatim}
   \bibliographystyle{ACM-Reference-Format}
   \bibliography{mybibfile}
\end{verbatim}
%
Here we assume that `\texttt{mybibfile.bib}' is the name of your 
\BibTeX\ file. Use the `\verb|\cite|' command to produce citations 
to your references. Here are a few examples for citations of journal 
articles~\cite{GrKr96,WoJe95}, books~\cite{Knu97}, articles in 
conference proceedings~\cite{Hag1993}, technical reports~\cite{Har78},
Master's and PhD theses~\cite{Ani03,Cla85}, online videos~\cite{Oba08}, 
datasets~\cite{AnMC13}, and patents~\cite{Sci09}. Both citations and 
references are numbered by default. 

Make sure you provide complete and correct bibliographic information 
for all your references, and list authors with their full names 
(``Donald E.\ Knuth'') rather than just initials (``D.\ E.\ Knuth''). 

%%%%%%%%%%%%%%%%%%%%%%%%%%%%%%%%%%%%%%%%%%%%%%%%%%%%%%%%%%%%%%%%%%%%%%%%

%%% The acknowledgments section is defined using the "acks" environment
%%% (rather than an unnumbered section). The use of this environment 
%%% ensures the proper identification of the section in the article 
%%% metadata as well as the consistent spelling of the heading.

\begin{acks}
If you wish to include any acknowledgments in your paper (e.g., to 
people or funding agencies), please do so using the `\texttt{acks}' 
environment. Note that the text of your acknowledgments will be omitted
if you compile your document with the `\texttt{anonymous}' option.
\end{acks}

%%%%%%%%%%%%%%%%%%%%%%%%%%%%%%%%%%%%%%%%%%%%%%%%%%%%%%%%%%%%%%%%%%%%%%%%

%%% The next two lines define, first, the bibliography style to be 
%%% applied, and, second, the bibliography file to be used.

\bibliographystyle{ACM-Reference-Format} 
\bibliography{sample}

%%%%%%%%%%%%%%%%%%%%%%%%%%%%%%%%%%%%%%%%%%%%%%%%%%%%%%%%%%%%%%%%%%%%%%%%

\end{document}

%%%%%%%%%%%%%%%%%%%%%%%%%%%%%%%%%%%%%%%%%%%%%%%%%%%%%%%%%%%%%%%%%%%%%%%%

